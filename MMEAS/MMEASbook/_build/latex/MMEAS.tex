%% Generated by Sphinx.
\def\sphinxdocclass{jupyterBook}
\documentclass[letterpaper,10pt,english]{jupyterBook}
\ifdefined\pdfpxdimen
   \let\sphinxpxdimen\pdfpxdimen\else\newdimen\sphinxpxdimen
\fi \sphinxpxdimen=.75bp\relax
%% turn off hyperref patch of \index as sphinx.xdy xindy module takes care of
%% suitable \hyperpage mark-up, working around hyperref-xindy incompatibility
\PassOptionsToPackage{hyperindex=false}{hyperref}
%% memoir class requires extra handling
\makeatletter\@ifclassloaded{memoir}
{\ifdefined\memhyperindexfalse\memhyperindexfalse\fi}{}\makeatother

\PassOptionsToPackage{warn}{textcomp}

\catcode`^^^^00a0\active\protected\def^^^^00a0{\leavevmode\nobreak\ }
\usepackage{cmap}
\usepackage{fontspec}
\defaultfontfeatures[\rmfamily,\sffamily,\ttfamily]{}
\usepackage{amsmath,amssymb,amstext}
\usepackage{polyglossia}
\setmainlanguage{english}



\setmainfont{FreeSerif}[
  Extension      = .otf,
  UprightFont    = *,
  ItalicFont     = *Italic,
  BoldFont       = *Bold,
  BoldItalicFont = *BoldItalic
]
\setsansfont{FreeSans}[
  Extension      = .otf,
  UprightFont    = *,
  ItalicFont     = *Oblique,
  BoldFont       = *Bold,
  BoldItalicFont = *BoldOblique,
]
\setmonofont{FreeMono}[
  Extension      = .otf,
  UprightFont    = *,
  ItalicFont     = *Oblique,
  BoldFont       = *Bold,
  BoldItalicFont = *BoldOblique,
]


\usepackage[Bjarne]{fncychap}
\usepackage[,numfigreset=1,mathnumfig]{sphinx}

\fvset{fontsize=\small}
\usepackage{geometry}


% Include hyperref last.
\usepackage{hyperref}
% Fix anchor placement for figures with captions.
\usepackage{hypcap}% it must be loaded after hyperref.
% Set up styles of URL: it should be placed after hyperref.
\urlstyle{same}


\usepackage{sphinxmessages}



        % Start of preamble defined in sphinx-jupyterbook-latex %
         \usepackage[Latin,Greek]{ucharclasses}
        \usepackage{unicode-math}
        % fixing title of the toc
        \addto\captionsenglish{\renewcommand{\contentsname}{Contents}}
        \hypersetup{
            pdfencoding=auto,
            psdextra
        }
        % End of preamble defined in sphinx-jupyterbook-latex %
        

\title{Mathematics Methods for Engineering and Applied Science}
\date{Oct 16, 2021}
\release{}
\author{Arseniy Buchnev}
\newcommand{\sphinxlogo}{\vbox{}}
\renewcommand{\releasename}{}
\makeindex
\begin{document}

\pagestyle{empty}
\sphinxmaketitle
\pagestyle{plain}
\sphinxtableofcontents
\pagestyle{normal}
\phantomsection\label{\detokenize{intro::doc}}


\sphinxAtStartPar
This webpage contains all assignments and solutions to the MMEAS course.


\chapter{Problem Set 1}
\label{\detokenize{notebooks/ProblemSet1:problem-set-1}}\label{\detokenize{notebooks/ProblemSet1::doc}}
\begin{sphinxVerbatim}[commandchars=\\\{\}]
\PYG{k+kn}{import} \PYG{n+nn}{numpy} \PYG{k}{as} \PYG{n+nn}{np}
\PYG{k+kn}{import} \PYG{n+nn}{matplotlib}\PYG{n+nn}{.}\PYG{n+nn}{pyplot} \PYG{k}{as} \PYG{n+nn}{plt}
\end{sphinxVerbatim}


\section{Problem 1}
\label{\detokenize{notebooks/ProblemSet1:problem-1}}

\subsection{Some basic problems on matrix/vector multiplication.}
\label{\detokenize{notebooks/ProblemSet1:some-basic-problems-on-matrix-vector-multiplication}}
\sphinxAtStartPar
(a) Calculate by hand the following matrix/vector products:
\begin{equation*}
\begin{split}
(i) \hspace{10mm}
\begin{bmatrix}
2 & 1\\
\end{bmatrix}
\begin{bmatrix}
1 & −2 & 1\\
2 & 3 & 2
\end{bmatrix};
\\
\\
(ii) \hspace{10mm}
\begin{bmatrix}
2 & -1 & 1\\
3 & 0 & 4
\end{bmatrix}
\begin{bmatrix}
2 & 1\\
1 & 2\\
-1 & 0
\end{bmatrix}
\end{split}
\end{equation*}
\sphinxAtStartPar
as a combination of columns of the left matrix aswell as a combination of rows of the right matrix.

\sphinxAtStartPar
(b) Write down a permutation matrix \(P_4\) that exchanges row 1 with row 3 and row 2with row 4. What is the connection of this matrix with the permutation matricesthat exchange only row 1 and row 3, and only row 2 and row 4?


\subsection{Solution}
\label{\detokenize{notebooks/ProblemSet1:solution}}
\noindent{\hspace*{\fill}\sphinxincludegraphics[width=800\sphinxpxdimen]{{problem1}.jpg}\hspace*{\fill}}


\section{Problem 2}
\label{\detokenize{notebooks/ProblemSet1:problem-2}}
\sphinxAtStartPar
Given a \(3\times3\) matrix \(A= \begin{bmatrix}a1&a2&a3\end{bmatrix}\) with columns \(a_i\), find a matrix \(B\) that when multiplied with \(A\), either from left or right, performs the following operations with \(A\):

\sphinxAtStartPar
(a) exchanges row 1 and row 2;(b) exchanges columns 1 and 2;(c) doubles the first row;(d) subtracts twice row 1 from row 2.Also find the inverse of this matrix. What doesthe inverse of this \(B\) do?


\subsection{Solution}
\label{\detokenize{notebooks/ProblemSet1:id1}}
\noindent{\hspace*{\fill}\sphinxincludegraphics[width=800\sphinxpxdimen]{{problem2_1}.jpg}\hspace*{\fill}}

\noindent{\hspace*{\fill}\sphinxincludegraphics[width=800\sphinxpxdimen]{{attachments/ProblemSet1_data/problem2(2)}.jpg}\hspace*{\fill}}


\section{Problem 3}
\label{\detokenize{notebooks/ProblemSet1:problem-3}}
\sphinxAtStartPar
For matrix
A =
\(\begin{bmatrix}
1 & 2 & 3\\
3 & 4 & 5\\
5 & 6 & 7
\end{bmatrix}\)
, determine the following:

\sphinxAtStartPar
(a) rank;(b) eigenvalues and eigenvectors;(c) nullspace and left nullspace;(d) column space and row space;(e) write \(A\) as a sum of rank\sphinxhyphen{}1 matrices in at least two different ways.


\subsection{Solution}
\label{\detokenize{notebooks/ProblemSet1:id2}}
\noindent{\hspace*{\fill}\sphinxincludegraphics[width=800\sphinxpxdimen]{{attachments/ProblemSet1_data/problem3(1)}.jpg}\hspace*{\fill}}

\noindent{\hspace*{\fill}\sphinxincludegraphics[width=800\sphinxpxdimen]{{attachments/ProblemSet1_data/problem3(2)}.jpg}\hspace*{\fill}}

\sphinxAtStartPar
\sphinxstyleemphasis{\sphinxstylestrong{d) Column space and row space:}}
\begin{itemize}
\item {} 
\sphinxAtStartPar
Column space: \(span(
\begin{bmatrix}
1\\
0\\
-1
\end{bmatrix},
\begin{bmatrix}
0\\
1\\
2
\end{bmatrix})
\); equals to the row space.

\end{itemize}

\sphinxAtStartPar
\sphinxstyleemphasis{\sphinxstylestrong{e) write \(A\)  as a sum of rank\sphinxhyphen{}1 matrices in at least two different ways:}}

\sphinxAtStartPar
Obviously, we can present the matrix \(A\) as a sum of single\sphinxhyphen{}row or single\sphinxhyphen{}column rank\sphinxhyphen{}1 matricies (with other rows or columns filled with zeros). Also, we have 3 non\sphinxhyphen{}colinear eigenvectors, that can be our matrix basis in the canonical decomposition (diagonalization):
\begin{equation*}
\begin{split}
A = UDU^{-1} = \sum\limits_{i=1}\limits^{3}{d_i\lambda_i\lambda_i^T}
\end{split}
\end{equation*}

\section{Problem 4}
\label{\detokenize{notebooks/ProblemSet1:problem-4}}
\sphinxAtStartPar
The columns of matrix \(C = 
\begin{bmatrix}
2 & 2 & 1 & 1 & 2 & 2 & 1 & 1\\
1 & 2 & 2 & 1 & 1 & 2 & 2 & 1\\
1 & 1 & 1 & 1 & 2 & 2 & 2 & 2
\end{bmatrix}
\)
represent vertices of a cube. Describe transformations of the cube that result from the action on \(C\) of the following three matrices:

\sphinxAtStartPar
\(A_1 = 
\begin{bmatrix}
1 & 2 & 2\\
0 & 2 & 2\\
0 & 0 & 3
\end{bmatrix}, \hspace{3mm}
A_2 = 
\begin{bmatrix}
1 & 2 & 2\\
0 & 2 & 2\\
0 & 0 & 0
\end{bmatrix}, \hspace{3mm}
A3 = 
\begin{bmatrix}
0 & 2 & 2\\
0 & 2 & 2\\
0 & 0 & 0
\end{bmatrix}.
\)

\sphinxAtStartPar
Relate the results to the ranks of \(A_k\) and to the dimensions and bases of the four fundamental subspaces of \(A_k\). Is there a \(3\times3\) matrix \(A\) that can transform a cube into a tetrahedron? Explain.


\subsection{Solution}
\label{\detokenize{notebooks/ProblemSet1:id3}}
\sphinxAtStartPar
Firstly, let us see what every transformation does with the cube.

\begin{sphinxVerbatim}[commandchars=\\\{\}]
\PYG{c+c1}{\PYGZsh{} Creating matricies}

\PYG{n}{A1} \PYG{o}{=} \PYG{n}{np}\PYG{o}{.}\PYG{n}{matrix}\PYG{p}{(}\PYG{p}{[}
    \PYG{p}{[}\PYG{l+m+mi}{1}\PYG{p}{,} \PYG{l+m+mi}{2}\PYG{p}{,} \PYG{l+m+mi}{2}\PYG{p}{]}\PYG{p}{,}
    \PYG{p}{[}\PYG{l+m+mi}{0}\PYG{p}{,} \PYG{l+m+mi}{2}\PYG{p}{,} \PYG{l+m+mi}{2}\PYG{p}{]}\PYG{p}{,}
    \PYG{p}{[}\PYG{l+m+mi}{0}\PYG{p}{,} \PYG{l+m+mi}{0}\PYG{p}{,} \PYG{l+m+mi}{3}\PYG{p}{]}
\PYG{p}{]}\PYG{p}{)}

\PYG{n}{A2} \PYG{o}{=} \PYG{n}{np}\PYG{o}{.}\PYG{n}{matrix}\PYG{p}{(}\PYG{p}{[}
    \PYG{p}{[}\PYG{l+m+mi}{1}\PYG{p}{,} \PYG{l+m+mi}{2}\PYG{p}{,} \PYG{l+m+mi}{2}\PYG{p}{]}\PYG{p}{,}
    \PYG{p}{[}\PYG{l+m+mi}{0}\PYG{p}{,} \PYG{l+m+mi}{2}\PYG{p}{,} \PYG{l+m+mi}{2}\PYG{p}{]}\PYG{p}{,}
    \PYG{p}{[}\PYG{l+m+mi}{0}\PYG{p}{,} \PYG{l+m+mi}{0}\PYG{p}{,} \PYG{l+m+mi}{0}\PYG{p}{]}
\PYG{p}{]}\PYG{p}{)}

\PYG{n}{A3} \PYG{o}{=} \PYG{n}{np}\PYG{o}{.}\PYG{n}{matrix}\PYG{p}{(}\PYG{p}{[}
    \PYG{p}{[}\PYG{l+m+mi}{0}\PYG{p}{,} \PYG{l+m+mi}{2}\PYG{p}{,} \PYG{l+m+mi}{2}\PYG{p}{]}\PYG{p}{,}
    \PYG{p}{[}\PYG{l+m+mi}{0}\PYG{p}{,} \PYG{l+m+mi}{2}\PYG{p}{,} \PYG{l+m+mi}{2}\PYG{p}{]}\PYG{p}{,}
    \PYG{p}{[}\PYG{l+m+mi}{0}\PYG{p}{,} \PYG{l+m+mi}{0}\PYG{p}{,} \PYG{l+m+mi}{0}\PYG{p}{]}
\PYG{p}{]}\PYG{p}{)}

\PYG{n}{C} \PYG{o}{=} \PYG{n}{np}\PYG{o}{.}\PYG{n}{matrix}\PYG{p}{(}\PYG{p}{[}
    \PYG{p}{[}\PYG{l+m+mi}{2}\PYG{p}{,} \PYG{l+m+mi}{2}\PYG{p}{,} \PYG{l+m+mi}{1}\PYG{p}{,} \PYG{l+m+mi}{1}\PYG{p}{,} \PYG{l+m+mi}{2}\PYG{p}{,} \PYG{l+m+mi}{2}\PYG{p}{,} \PYG{l+m+mi}{1}\PYG{p}{,} \PYG{l+m+mi}{1}\PYG{p}{]}\PYG{p}{,}
    \PYG{p}{[}\PYG{l+m+mi}{1}\PYG{p}{,} \PYG{l+m+mi}{2}\PYG{p}{,} \PYG{l+m+mi}{2}\PYG{p}{,} \PYG{l+m+mi}{1}\PYG{p}{,} \PYG{l+m+mi}{1}\PYG{p}{,} \PYG{l+m+mi}{2}\PYG{p}{,} \PYG{l+m+mi}{2}\PYG{p}{,} \PYG{l+m+mi}{1}\PYG{p}{]}\PYG{p}{,}
    \PYG{p}{[}\PYG{l+m+mi}{1}\PYG{p}{,} \PYG{l+m+mi}{1}\PYG{p}{,} \PYG{l+m+mi}{1}\PYG{p}{,} \PYG{l+m+mi}{1}\PYG{p}{,} \PYG{l+m+mi}{2}\PYG{p}{,} \PYG{l+m+mi}{2}\PYG{p}{,} \PYG{l+m+mi}{2}\PYG{p}{,} \PYG{l+m+mi}{2}\PYG{p}{]}
\PYG{p}{]}\PYG{p}{)}

\PYG{c+c1}{\PYGZsh{}Applying A matricies to the cube}

\PYG{n}{B1} \PYG{o}{=} \PYG{n}{A1}\PYG{o}{*}\PYG{n}{C}
\PYG{n}{B2} \PYG{o}{=} \PYG{n}{A2}\PYG{o}{*}\PYG{n}{C}
\PYG{n}{B3} \PYG{o}{=} \PYG{n}{A3}\PYG{o}{*}\PYG{n}{C}

\PYG{c+c1}{\PYGZsh{}Plotting}

\PYG{o}{\PYGZpc{}}\PYG{k}{config} InlineBackend.figure\PYGZus{}format = \PYGZsq{}svg\PYGZsq{}
\PYG{n}{fig} \PYG{o}{=} \PYG{n}{plt}\PYG{o}{.}\PYG{n}{figure}\PYG{p}{(}\PYG{p}{)}
\PYG{n}{ax} \PYG{o}{=} \PYG{n}{fig}\PYG{o}{.}\PYG{n}{add\PYGZus{}subplot}\PYG{p}{(}\PYG{l+m+mi}{111}\PYG{p}{,} \PYG{n}{projection}\PYG{o}{=}\PYG{l+s+s1}{\PYGZsq{}}\PYG{l+s+s1}{3d}\PYG{l+s+s1}{\PYGZsq{}}\PYG{p}{)}
\PYG{n}{ax}\PYG{o}{.}\PYG{n}{scatter}\PYG{p}{(}\PYG{n}{xs} \PYG{o}{=} \PYG{n}{C}\PYG{p}{[}\PYG{l+m+mi}{0}\PYG{p}{]}\PYG{p}{,} \PYG{n}{ys} \PYG{o}{=} \PYG{n}{C}\PYG{p}{[}\PYG{l+m+mi}{1}\PYG{p}{]}\PYG{p}{,} \PYG{n}{zs} \PYG{o}{=} \PYG{n}{C}\PYG{p}{[}\PYG{l+m+mi}{2}\PYG{p}{]}\PYG{p}{)}
\PYG{n}{ax}\PYG{o}{.}\PYG{n}{scatter}\PYG{p}{(}\PYG{n}{xs} \PYG{o}{=} \PYG{n}{B1}\PYG{p}{[}\PYG{l+m+mi}{0}\PYG{p}{]}\PYG{p}{,} \PYG{n}{ys} \PYG{o}{=} \PYG{n}{B1}\PYG{p}{[}\PYG{l+m+mi}{1}\PYG{p}{]}\PYG{p}{,} \PYG{n}{zs} \PYG{o}{=} \PYG{n}{B1}\PYG{p}{[}\PYG{l+m+mi}{2}\PYG{p}{]}\PYG{p}{,} \PYG{n}{color}\PYG{o}{=}\PYG{l+s+s1}{\PYGZsq{}}\PYG{l+s+s1}{red}\PYG{l+s+s1}{\PYGZsq{}}\PYG{p}{,} \PYG{n}{depthshade} \PYG{o}{=} \PYG{k+kc}{True}\PYG{p}{)}
\PYG{n}{ax}\PYG{o}{.}\PYG{n}{scatter}\PYG{p}{(}\PYG{n}{xs} \PYG{o}{=} \PYG{n}{B2}\PYG{p}{[}\PYG{l+m+mi}{0}\PYG{p}{]}\PYG{p}{,} \PYG{n}{ys} \PYG{o}{=} \PYG{n}{B2}\PYG{p}{[}\PYG{l+m+mi}{1}\PYG{p}{]}\PYG{p}{,} \PYG{n}{zs} \PYG{o}{=} \PYG{n}{B2}\PYG{p}{[}\PYG{l+m+mi}{2}\PYG{p}{]}\PYG{p}{,} \PYG{n}{color}\PYG{o}{=}\PYG{l+s+s1}{\PYGZsq{}}\PYG{l+s+s1}{green}\PYG{l+s+s1}{\PYGZsq{}}\PYG{p}{,} \PYG{n}{depthshade} \PYG{o}{=} \PYG{k+kc}{True}\PYG{p}{)}
\PYG{n}{ax}\PYG{o}{.}\PYG{n}{scatter}\PYG{p}{(}\PYG{n}{xs} \PYG{o}{=} \PYG{n}{B3}\PYG{p}{[}\PYG{l+m+mi}{0}\PYG{p}{]}\PYG{p}{,} \PYG{n}{ys} \PYG{o}{=} \PYG{n}{B3}\PYG{p}{[}\PYG{l+m+mi}{1}\PYG{p}{]}\PYG{p}{,} \PYG{n}{zs} \PYG{o}{=} \PYG{n}{B3}\PYG{p}{[}\PYG{l+m+mi}{2}\PYG{p}{]}\PYG{p}{,} \PYG{n}{color}\PYG{o}{=}\PYG{l+s+s1}{\PYGZsq{}}\PYG{l+s+s1}{yellow}\PYG{l+s+s1}{\PYGZsq{}}\PYG{p}{,} \PYG{n}{depthshade} \PYG{o}{=} \PYG{k+kc}{True}\PYG{p}{)}
\PYG{n}{ax}\PYG{o}{.}\PYG{n}{set\PYGZus{}xticks}\PYG{p}{(}\PYG{n}{np}\PYG{o}{.}\PYG{n}{arange}\PYG{p}{(}\PYG{l+m+mi}{0}\PYG{p}{,} \PYG{l+m+mi}{11}\PYG{p}{,} \PYG{l+m+mi}{2}\PYG{p}{)}\PYG{p}{)}
\PYG{n}{ax}\PYG{o}{.}\PYG{n}{set\PYGZus{}yticks}\PYG{p}{(}\PYG{n}{np}\PYG{o}{.}\PYG{n}{arange}\PYG{p}{(}\PYG{l+m+mi}{0}\PYG{p}{,} \PYG{l+m+mi}{11}\PYG{p}{,} \PYG{l+m+mi}{2}\PYG{p}{)}\PYG{p}{)}
\PYG{n}{ax}\PYG{o}{.}\PYG{n}{set\PYGZus{}zticks}\PYG{p}{(}\PYG{n}{np}\PYG{o}{.}\PYG{n}{arange}\PYG{p}{(}\PYG{l+m+mi}{0}\PYG{p}{,} \PYG{l+m+mi}{7}\PYG{p}{,} \PYG{l+m+mi}{1}\PYG{p}{)}\PYG{p}{)}
\PYG{n}{ax}\PYG{o}{.}\PYG{n}{set\PYGZus{}xlabel}\PYG{p}{(}\PYG{l+s+s1}{\PYGZsq{}}\PYG{l+s+s1}{x}\PYG{l+s+s1}{\PYGZsq{}}\PYG{p}{)}
\PYG{n}{ax}\PYG{o}{.}\PYG{n}{set\PYGZus{}ylabel}\PYG{p}{(}\PYG{l+s+s1}{\PYGZsq{}}\PYG{l+s+s1}{y}\PYG{l+s+s1}{\PYGZsq{}}\PYG{p}{)}
\PYG{n}{ax}\PYG{o}{.}\PYG{n}{set\PYGZus{}zlabel}\PYG{p}{(}\PYG{l+s+s1}{\PYGZsq{}}\PYG{l+s+s1}{z}\PYG{l+s+s1}{\PYGZsq{}}\PYG{p}{)}

\PYG{n}{plt}\PYG{o}{.}\PYG{n}{title}\PYG{p}{(}\PYG{l+s+s1}{\PYGZsq{}}\PYG{l+s+s1}{Cube Transformations}\PYG{l+s+s1}{\PYGZsq{}}\PYG{p}{)}

\PYG{n}{plt}\PYG{o}{.}\PYG{n}{legend}\PYG{p}{(}\PYG{p}{(}\PYG{l+s+s1}{\PYGZsq{}}\PYG{l+s+s1}{Initial Cube}\PYG{l+s+s1}{\PYGZsq{}}\PYG{p}{,} \PYG{l+s+s1}{\PYGZsq{}}\PYG{l+s+s1}{A1}\PYG{l+s+s1}{\PYGZsq{}}\PYG{p}{,} \PYG{l+s+s1}{\PYGZsq{}}\PYG{l+s+s1}{A2}\PYG{l+s+s1}{\PYGZsq{}}\PYG{p}{,} \PYG{l+s+s1}{\PYGZsq{}}\PYG{l+s+s1}{A3}\PYG{l+s+s1}{\PYGZsq{}}\PYG{p}{)}\PYG{p}{)}
\end{sphinxVerbatim}

\begin{sphinxVerbatim}[commandchars=\\\{\}]
\PYGZlt{}matplotlib.legend.Legend at 0x25bffd95808\PYGZgt{}
\end{sphinxVerbatim}

\begin{sphinxVerbatim}[commandchars=\\\{\}]
\PYGZlt{}Figure size 432x288 with 1 Axes\PYGZgt{}
\end{sphinxVerbatim}

\sphinxAtStartPar
The points on this 3D plot represent the verticies of the initial cube (blue) and resulting shapes.
As we see,
\begin{itemize}
\item {} 
\sphinxAtStartPar
A1 combines some stretching and rotation;

\item {} 
\sphinxAtStartPar
A2 projects the first transformation on the \(XY\) plane;

\item {} 
\sphinxAtStartPar
A3 projects the first (or the second) transformation on a line.

\end{itemize}

\sphinxAtStartPar
Now, let’s derive the properties of each matrix \(A_i\).
\begin{equation*}
\begin{split}
rg(A_1) = 3;\hspace{3mm}rg(A_2) = 2;\hspace{3mm}rg(A_3) = 1
\end{split}
\end{equation*}
\sphinxAtStartPar
It follows from the Main theorem of Linear Algebra that
\begin{equation*}
\begin{split}
rg(ker(A_1)) = 0;\hspace{3mm}rg(ker(A_2)) = 1;\hspace{3mm}rg(ker(A_3)) = 2
\end{split}
\end{equation*}
\sphinxAtStartPar
That means,
\begin{itemize}
\item {} 
\sphinxAtStartPar
\(A_1\) maps \(\mathbb{R}^3\) to \(\mathbb{R}^3\)

\item {} 
\sphinxAtStartPar
\(A_2\) maps \(\mathbb{R}^3\) to \(\mathbb{R}^2\)

\item {} 
\sphinxAtStartPar
\(A_3\) maps \(\mathbb{R}^3\) to \(\mathbb{R}^1\)

\end{itemize}

\sphinxAtStartPar
That corresponds to our findings in the previous section.

\sphinxAtStartPar
Let us find the four fundamental subspaces for each \(A_i\) operator:
\begin{itemize}
\item {} 
\sphinxAtStartPar
\(A_1\) is a full\sphinxhyphen{}rank matrix, that means nullspace is empty.

\end{itemize}

\sphinxAtStartPar
The same applies to \(A_1^T\). \(A_1\) column space is \(
\begin{bmatrix}
1\\
0\\
0
\end{bmatrix}
,
\begin{bmatrix}
0\\
2\\
0
\end{bmatrix},
\begin{bmatrix}
0\\
0\\
3
\end{bmatrix}
\), as well as for \(A_1^T\) (row space of \(A_1\)).
\begin{itemize}
\item {} 
\sphinxAtStartPar
for \(A_2\) we solve a simple set of linear equations and receive the following result:

\end{itemize}
\begin{equation*}
\begin{split}Null(A_2) = span(
\begin{bmatrix}
0\\
1\\
-1
\end{bmatrix}
)
; \hspace{3mm} Null(A_2^T) = span(
\begin{bmatrix}
0\\
1\\
-1
\end{bmatrix}
).
\end{split}
\end{equation*}
\sphinxAtStartPar
The column space of \(A_2\) is
\(
\begin{bmatrix}
1\\
0\\
0
\end{bmatrix},
\begin{bmatrix}
0\\
2\\
0
\end{bmatrix}
\), the row space is the same.
\begin{itemize}
\item {} 
\sphinxAtStartPar
for \(A_3\), we have two\sphinxhyphen{}dimensional null space: \(Null(A_3) = span(
\begin{bmatrix}
1\\
1\\
-1
\end{bmatrix},
\begin{bmatrix}
1\\
-1\\
1
\end{bmatrix}
)\). It coincides with the nullspace of \(A_3^T\).

\end{itemize}

\sphinxAtStartPar
The column space of \(A_3\) is \(span(
\begin{bmatrix}
1\\
1\\
0
\end{bmatrix})\), as well as the row space. This result is easily observed on the visualisation: the result of the 3rd transformation lies on the \(y = x\) line on the \(XY\) plane.

\begin{sphinxVerbatim}[commandchars=\\\{\}]
\PYG{k}{def} \PYG{n+nf}{print\PYGZus{}eig}\PYG{p}{(}\PYG{n}{A}\PYG{p}{,} \PYG{n}{number}\PYG{p}{)}\PYG{p}{:}
    \PYG{n}{val}\PYG{p}{,} \PYG{n}{vec} \PYG{o}{=} \PYG{n}{np}\PYG{o}{.}\PYG{n}{linalg}\PYG{o}{.}\PYG{n}{eig}\PYG{p}{(}\PYG{n}{A}\PYG{p}{)}
    \PYG{n+nb}{print}\PYG{p}{(}\PYG{l+s+s2}{\PYGZdq{}}\PYG{l+s+s2}{Eigenvalues and eigenvectors for matrix A}\PYG{l+s+si}{\PYGZob{}\PYGZcb{}}\PYG{l+s+s2}{:}\PYG{l+s+s2}{\PYGZdq{}}\PYG{o}{.}\PYG{n}{format}\PYG{p}{(}\PYG{n}{number}\PYG{p}{)}\PYG{p}{)}
    \PYG{n+nb}{print}\PYG{p}{(}\PYG{n}{val}\PYG{p}{)}
    \PYG{n+nb}{print}\PYG{p}{(}\PYG{n}{vec}\PYG{p}{)}
    \PYG{n+nb}{print}\PYG{p}{(}\PYG{p}{)}
    
\PYG{c+c1}{\PYGZsh{}print\PYGZus{}eig(A1, 1)}
\PYG{c+c1}{\PYGZsh{}print\PYGZus{}eig(A2, 2)}
\PYG{c+c1}{\PYGZsh{}print\PYGZus{}eig(A3, 3)}
\end{sphinxVerbatim}

\sphinxAtStartPar
\sphinxstylestrong{On transforming a cube into a tetrahedron:}
Linear transformation implies that we can describe it only acting on basis vectors.
That means, I suppose there is no such \(3\times 3\) matrix to perform this operation.


\section{Problem 5}
\label{\detokenize{notebooks/ProblemSet1:problem-5}}
\sphinxAtStartPar
For matrix
\(A=
\begin{bmatrix}
2 & 1\\
1 & 2
\end{bmatrix}
\)
determine which unit vector \(x_M\) is stretched the most and which \(x_m\) the least and by how much. That is, find \(x\) such that \(y=Ax\) has the largest (or smallest) possible Euclidian length. You can do this by calculus methods, e.g. using Lagrange multipliers. Relate your findings to eigenvalues and eigenvectors of \(A\).


\subsection{Solution}
\label{\detokenize{notebooks/ProblemSet1:id4}}
\sphinxAtStartPar
Firstly, we SVD the matrix \(A\):

\begin{sphinxVerbatim}[commandchars=\\\{\}]
\PYG{n}{A} \PYG{o}{=} \PYG{n}{np}\PYG{o}{.}\PYG{n}{matrix}\PYG{p}{(}\PYG{p}{[}
    \PYG{p}{[}\PYG{l+m+mi}{2}\PYG{p}{,} \PYG{l+m+mi}{1}\PYG{p}{]}\PYG{p}{,}
    \PYG{p}{[}\PYG{l+m+mi}{1}\PYG{p}{,} \PYG{l+m+mi}{2}\PYG{p}{]}
\PYG{p}{]}\PYG{p}{)}

\PYG{n}{u}\PYG{p}{,} \PYG{n}{s}\PYG{p}{,} \PYG{n}{v} \PYG{o}{=} \PYG{n}{np}\PYG{o}{.}\PYG{n}{linalg}\PYG{o}{.}\PYG{n}{svd}\PYG{p}{(}\PYG{n}{A}\PYG{p}{)}
\PYG{n+nb}{print}\PYG{p}{(}\PYG{l+s+s1}{\PYGZsq{}}\PYG{l+s+s1}{U:}\PYG{l+s+se}{\PYGZbs{}n}\PYG{l+s+s1}{\PYGZsq{}}\PYG{p}{,} \PYG{n}{u}\PYG{p}{)}
\PYG{n+nb}{print}\PYG{p}{(}\PYG{l+s+s1}{\PYGZsq{}}\PYG{l+s+s1}{E:}\PYG{l+s+s1}{\PYGZsq{}}\PYG{p}{,} \PYG{n}{s}\PYG{p}{)}
\PYG{n+nb}{print}\PYG{p}{(}\PYG{l+s+s1}{\PYGZsq{}}\PYG{l+s+s1}{V*:}\PYG{l+s+se}{\PYGZbs{}n}\PYG{l+s+s1}{\PYGZsq{}}\PYG{p}{,} \PYG{n}{v}\PYG{p}{)}
\end{sphinxVerbatim}

\begin{sphinxVerbatim}[commandchars=\\\{\}]
U:
 [[\PYGZhy{}0.70710678 \PYGZhy{}0.70710678]
 [\PYGZhy{}0.70710678  0.70710678]]
E: [3. 1.]
V*:
 [[\PYGZhy{}0.70710678 \PYGZhy{}0.70710678]
 [\PYGZhy{}0.70710678  0.70710678]]
\end{sphinxVerbatim}

\sphinxAtStartPar
We see that the matricies \(U\) and \(V*\) may be rewritten as a combination of the following matricies:
\begin{equation*}
\begin{split}
U = V^* = 
\begin{bmatrix}
-\frac{\sqrt{2}}{2} & -\frac{\sqrt{2}}{2}\\
-\frac{\sqrt{2}}{2} & \frac{\sqrt{2}}{2}
\end{bmatrix}
=
\frac{\sqrt{2}}{2}
\begin{bmatrix}
-1 & -1\\
-1 & 1
\end{bmatrix}
=
\frac{\sqrt{2}}{2}
\begin{bmatrix}
1 & -1\\
1 & 1
\end{bmatrix}
\times
\begin{bmatrix}
-1 & 0\\
0 & 1
\end{bmatrix}
=\\
=cR\times M,
\end{split}
\end{equation*}
\sphinxAtStartPar
where \(c = \frac{\sqrt{2}}{2}\), \(cR\) is a rotation matrix with \(\varphi = \frac{1}{4}\pi\), \(M\) is a mirror operator that mirrors about \(X\) axis.

\sphinxAtStartPar
The matrix \(\Sigma\) has diagonal elements \(\begin{bmatrix} 3 & 1 \end{bmatrix}\) and represents stretching times 3 along \(X\) axis.

\begin{sphinxVerbatim}[commandchars=\\\{\}]
\PYG{n}{M} \PYG{o}{=} \PYG{n}{np}\PYG{o}{.}\PYG{n}{matrix}\PYG{p}{(}\PYG{p}{[}
    \PYG{p}{[}\PYG{o}{\PYGZhy{}}\PYG{l+m+mi}{1}\PYG{p}{,} \PYG{l+m+mi}{0}\PYG{p}{]}\PYG{p}{,}
    \PYG{p}{[}\PYG{l+m+mi}{0}\PYG{p}{,} \PYG{l+m+mi}{1}\PYG{p}{]}
\PYG{p}{]}\PYG{p}{)}
\PYG{n}{c} \PYG{o}{=} \PYG{n}{np}\PYG{o}{.}\PYG{n}{sqrt}\PYG{p}{(}\PYG{l+m+mi}{2}\PYG{p}{)}\PYG{o}{/}\PYG{l+m+mi}{2}
\PYG{n}{R} \PYG{o}{=} \PYG{n}{np}\PYG{o}{.}\PYG{n}{matrix}\PYG{p}{(}\PYG{p}{[}
    \PYG{p}{[}\PYG{l+m+mi}{1}\PYG{p}{,} \PYG{o}{\PYGZhy{}}\PYG{l+m+mi}{1}\PYG{p}{]}\PYG{p}{,}
    \PYG{p}{[}\PYG{l+m+mi}{1}\PYG{p}{,} \PYG{l+m+mi}{1}\PYG{p}{]}
\PYG{p}{]}\PYG{p}{)}

\PYG{c+c1}{\PYGZsh{}c*R*M}

\PYG{c+c1}{\PYGZsh{}c*R}
\end{sphinxVerbatim}

\sphinxAtStartPar
After the stretch has been completed, the matrix \(U\) does the reverse transformation: mirrors the \(X'\) axis and rotates \(\frac{1}{4}\pi\) counterclockwise.

\sphinxAtStartPar
Now we can say that the most stretched vector \(x_M\) will be the
\(
\frac{\sqrt{2}}{2}
\begin{bmatrix}
1\\
1
\end{bmatrix}
\) (or it’s mirrored counterpart)
unit vector, which during the transformations is aligned along the direction of the stretch operation and it’s Euclidian length will be \(3\), and the least stretched vector \(x_m\) is the
\(
\frac{\sqrt{2}}{2}
\begin{bmatrix}
1\\
-1
\end{bmatrix}
\) (or it’s mirrored counterpart), which lies perpendicular to the axis of the stretch operation. It’s Euclidian length remains \(1\).

\sphinxAtStartPar
Now let’s look at the Eigenvalues and Eigenvectors of the matrix (the values and the vectors can be easily calculated by hand):

\begin{sphinxVerbatim}[commandchars=\\\{\}]
\PYG{n}{val}\PYG{p}{,} \PYG{n}{vec} \PYG{o}{=} \PYG{n}{np}\PYG{o}{.}\PYG{n}{linalg}\PYG{o}{.}\PYG{n}{eig}\PYG{p}{(}\PYG{n}{A}\PYG{p}{)}
\PYG{n+nb}{print}\PYG{p}{(}\PYG{l+s+s1}{\PYGZsq{}}\PYG{l+s+s1}{Eigenvalues: }\PYG{l+s+s1}{\PYGZsq{}}\PYG{p}{,} \PYG{n}{val}\PYG{p}{)}
\PYG{n+nb}{print}\PYG{p}{(}\PYG{l+s+s1}{\PYGZsq{}}\PYG{l+s+s1}{Eigenvectors:}\PYG{l+s+s1}{\PYGZsq{}}\PYG{p}{)}
\PYG{n+nb}{print}\PYG{p}{(}\PYG{n}{vec}\PYG{p}{)}
\end{sphinxVerbatim}

\begin{sphinxVerbatim}[commandchars=\\\{\}]
Eigenvalues:  [3. 1.]
Eigenvectors:
[[ 0.70710678 \PYGZhy{}0.70710678]
 [ 0.70710678  0.70710678]]
\end{sphinxVerbatim}

\sphinxAtStartPar
The result coincides with the result derived from SV decomposition: the
\(
\frac{\sqrt{2}}{2}
\begin{bmatrix}
1\\
1
\end{bmatrix}
\)
vector is stretched 3 times, the
\(
\frac{\sqrt{2}}{2}
\begin{bmatrix}
1\\
-1
\end{bmatrix}
\)
vector remains the same.


\section{Problem 6}
\label{\detokenize{notebooks/ProblemSet1:problem-6}}
\sphinxAtStartPar
Find eigenvalues and eigenvectors of the following matrices:(a)
\(A_1=
\begin{bmatrix}
0 & 1\\
-1 & 0
\end{bmatrix}
\)
. If \(x\) is any real vector, how is \(y=A_1x\) related to \(x\) geometrically?

\sphinxAtStartPar
(b)
\(A_2=
\begin{bmatrix}
1 & 1 & 0\\
0 & 1 & 1\\
0 & 0 & 1
\end{bmatrix}
\)
. What is the rank of \(A_2\)? How many eigenvectors are there?


\subsection{Solution}
\label{\detokenize{notebooks/ProblemSet1:id5}}
\sphinxAtStartPar
The calculation of eigenvalues and eigenvectors may be performed by hand easily, as well as with \sphinxstyleemphasis{NumPy} package.

\sphinxAtStartPar
We start with matrix \(A_1\):

\begin{sphinxVerbatim}[commandchars=\\\{\}]
\PYG{n}{A1} \PYG{o}{=} \PYG{n}{np}\PYG{o}{.}\PYG{n}{matrix}\PYG{p}{(}\PYG{p}{[}
    \PYG{p}{[}\PYG{l+m+mi}{0}\PYG{p}{,} \PYG{l+m+mi}{1}\PYG{p}{]}\PYG{p}{,}
    \PYG{p}{[}\PYG{o}{\PYGZhy{}}\PYG{l+m+mi}{1}\PYG{p}{,} \PYG{l+m+mi}{0}\PYG{p}{]}
\PYG{p}{]}\PYG{p}{)}

\PYG{n}{A2} \PYG{o}{=} \PYG{n}{np}\PYG{o}{.}\PYG{n}{matrix}\PYG{p}{(}\PYG{p}{[}
    \PYG{p}{[}\PYG{l+m+mi}{1}\PYG{p}{,} \PYG{l+m+mi}{1}\PYG{p}{,} \PYG{l+m+mi}{0}\PYG{p}{]}\PYG{p}{,}
    \PYG{p}{[}\PYG{l+m+mi}{0}\PYG{p}{,} \PYG{l+m+mi}{1}\PYG{p}{,} \PYG{l+m+mi}{1}\PYG{p}{]}\PYG{p}{,}
    \PYG{p}{[}\PYG{l+m+mi}{0}\PYG{p}{,} \PYG{l+m+mi}{0}\PYG{p}{,} \PYG{l+m+mi}{1}\PYG{p}{]}
\PYG{p}{]}\PYG{p}{)}

\PYG{n}{val1}\PYG{p}{,} \PYG{n}{vec1} \PYG{o}{=} \PYG{n}{np}\PYG{o}{.}\PYG{n}{linalg}\PYG{o}{.}\PYG{n}{eig}\PYG{p}{(}\PYG{n}{A1}\PYG{p}{)}
\PYG{n}{val2}\PYG{p}{,} \PYG{n}{vec2} \PYG{o}{=} \PYG{n}{np}\PYG{o}{.}\PYG{n}{linalg}\PYG{o}{.}\PYG{n}{eig}\PYG{p}{(}\PYG{n}{A2}\PYG{p}{)}

\PYG{n}{print\PYGZus{}eig}\PYG{p}{(}\PYG{n}{A1}\PYG{p}{,} \PYG{l+m+mi}{1}\PYG{p}{)}
\end{sphinxVerbatim}

\begin{sphinxVerbatim}[commandchars=\\\{\}]
Eigenvalues and eigenvectors for matrix A1:
[0.+1.j 0.\PYGZhy{}1.j]
[[0.70710678+0.j         0.70710678\PYGZhy{}0.j        ]
 [0.        +0.70710678j 0.        \PYGZhy{}0.70710678j]]
\end{sphinxVerbatim}

\sphinxAtStartPar
Instantly from one glance at the matrix \(A_1\), as well as by seeing the result of eigen\_operations, we derive that the matrix conducts a rotation. There are no real eigenvectors, obviously, as there are no vectors \(x\) that would be collinearly translated into another vectors \(y\).

\sphinxAtStartPar
Now, let’s analyze the matrix \(A_2\). It is obvious, that \(rg(A_2) = 1\), because we can perform simple row operations (\(r_2 = r_2 - r_3; \hspace{3mm} r_1 = r_1 - (r_2 - r_3)\)) to get the resulting matrix \(
\begin{bmatrix}
1 & 0 & 0\\
0 & 0 & 0\\
0 & 0 & 0
\end{bmatrix}
\).

\begin{sphinxVerbatim}[commandchars=\\\{\}]
Eigenvalues and eigenvectors for matrix A2:
[1. 1. 1.]
[[ 1.00000000e+00 \PYGZhy{}1.00000000e+00  1.00000000e+00]
 [ 0.00000000e+00  2.22044605e\PYGZhy{}16 \PYGZhy{}2.22044605e\PYGZhy{}16]
 [ 0.00000000e+00  0.00000000e+00  4.93038066e\PYGZhy{}32]]
\end{sphinxVerbatim}

\sphinxAtStartPar
There is one eigenvalue \(\lambda = 1\) of multiplicity 3, and only one eigenvector
\(h = 
\begin{bmatrix}
1\\
0\\
0
\end{bmatrix}
\)
This matrix is a linear operator that projects on \(YZ\) plane.


\chapter{Problem Set 2}
\label{\detokenize{notebooks/ProblemSet2:problem-set-2}}\label{\detokenize{notebooks/ProblemSet2::doc}}

\section{Problem 1}
\label{\detokenize{notebooks/ProblemSet2:problem-1}}
\sphinxAtStartPar
Consider linear system
\begin{equation*}
\begin{split}
\begin{align*}
2x_1 + x_2 = 1\\
x_1 + 2x_2 + x_3 = 2\\
x_2 + 2x_3 = 3
\end{align*}
\end{split}
\end{equation*}
\sphinxAtStartPar
(a) Find the \(LU\) factorization of the coefficient matrix \(A\).   Show that \(U = DL^T\) with \(D\) diagonal and thus \(A=LDL^T\). Find the exact solution using the \(LU\) factorization.

\sphinxAtStartPar
(b) Solve the system using Jacobi and Gauss\sphinxhyphen{}Seidel iterations. How many iterations are needed to reduce the relative error of the solution to \(10^{-8}\)?

\sphinxAtStartPar
(c) Plot in semilog scales the relative errors by both methods as a function of the number of iterations.

\sphinxAtStartPar
(d) Explain the convergence rate. Which of the methods is better and why?


\subsection{Solution}
\label{\detokenize{notebooks/ProblemSet2:solution}}
\sphinxAtStartPar
Let us perform the \(LU\) factorization by hand:
\begin{equation*}
\begin{split}
\begin{align*}
\begin{bmatrix}
2 & 1 & 0\\
1 & 2 & 1\\
0 & 1 & 2
\end{bmatrix}
\Rightarrow (R_2 - \frac{1}{2}R_1)
\begin{bmatrix}
2 & 1 & 0\\
0 & \frac{3}{2} & 1\\
0 & 1 & 2
\end{bmatrix}
\Rightarrow (R_3 - \frac{2}{3}R_2)
\begin{bmatrix}
2 & 1 & 0\\
0 & \frac{3}{2} & 1\\
0 & 0 & \frac{4}{3}
\end{bmatrix}
= U;
\hspace{10mm}
L =
\begin{bmatrix}
1 & 0 & 0\\
\frac{1}{2} & 1 & 0\\
0 & \frac{2}{3} & 1
\end{bmatrix}
\end{align*}
\end{split}
\end{equation*}
\sphinxAtStartPar
And check it via \sphinxstyleemphasis{numpy}:

\begin{sphinxVerbatim}[commandchars=\\\{\}]
\PYG{k+kn}{import} \PYG{n+nn}{numpy} \PYG{k}{as} \PYG{n+nn}{np}
\PYG{k+kn}{from} \PYG{n+nn}{fractions} \PYG{k+kn}{import} \PYG{n}{Fraction}
\PYG{k+kn}{from} \PYG{n+nn}{scipy} \PYG{k+kn}{import} \PYG{n}{linalg} \PYG{k}{as} \PYG{n}{lin}
\PYG{k+kn}{import} \PYG{n+nn}{sympy} \PYG{k}{as} \PYG{n+nn}{sp}
\PYG{k+kn}{from} \PYG{n+nn}{sympy} \PYG{k+kn}{import} \PYG{n}{abc}

\PYG{n}{A} \PYG{o}{=} \PYG{n}{np}\PYG{o}{.}\PYG{n}{matrix}\PYG{p}{(}\PYG{p}{[}
    \PYG{p}{[}\PYG{l+m+mi}{2}\PYG{p}{,} \PYG{l+m+mi}{1}\PYG{p}{,} \PYG{l+m+mi}{0}\PYG{p}{]}\PYG{p}{,}
    \PYG{p}{[}\PYG{l+m+mi}{1}\PYG{p}{,} \PYG{l+m+mi}{2}\PYG{p}{,} \PYG{l+m+mi}{1}\PYG{p}{]}\PYG{p}{,}
    \PYG{p}{[}\PYG{l+m+mi}{0}\PYG{p}{,} \PYG{l+m+mi}{1}\PYG{p}{,} \PYG{l+m+mi}{2}\PYG{p}{]}
\PYG{p}{]}\PYG{p}{)}

\PYG{k}{def} \PYG{n+nf}{bmatrix}\PYG{p}{(}\PYG{n}{a}\PYG{p}{)}\PYG{p}{:}
    \PYG{l+s+sd}{\PYGZdq{}\PYGZdq{}\PYGZdq{}Returns a LaTeX bmatrix}

\PYG{l+s+sd}{    :a: numpy array}
\PYG{l+s+sd}{    :returns: LaTeX bmatrix as a string}
\PYG{l+s+sd}{    \PYGZdq{}\PYGZdq{}\PYGZdq{}}
    \PYG{k}{if} \PYG{n+nb}{len}\PYG{p}{(}\PYG{n}{a}\PYG{o}{.}\PYG{n}{shape}\PYG{p}{)} \PYG{o}{\PYGZgt{}} \PYG{l+m+mi}{2}\PYG{p}{:}
        \PYG{k}{raise} \PYG{n+ne}{ValueError}\PYG{p}{(}\PYG{l+s+s1}{\PYGZsq{}}\PYG{l+s+s1}{bmatrix can at most display two dimensions}\PYG{l+s+s1}{\PYGZsq{}}\PYG{p}{)}
    \PYG{n}{lines} \PYG{o}{=} \PYG{n+nb}{str}\PYG{p}{(}\PYG{n}{a}\PYG{p}{)}\PYG{o}{.}\PYG{n}{replace}\PYG{p}{(}\PYG{l+s+s1}{\PYGZsq{}}\PYG{l+s+s1}{[}\PYG{l+s+s1}{\PYGZsq{}}\PYG{p}{,} \PYG{l+s+s1}{\PYGZsq{}}\PYG{l+s+s1}{\PYGZsq{}}\PYG{p}{)}\PYG{o}{.}\PYG{n}{replace}\PYG{p}{(}\PYG{l+s+s1}{\PYGZsq{}}\PYG{l+s+s1}{]}\PYG{l+s+s1}{\PYGZsq{}}\PYG{p}{,} \PYG{l+s+s1}{\PYGZsq{}}\PYG{l+s+s1}{\PYGZsq{}}\PYG{p}{)}\PYG{o}{.}\PYG{n}{splitlines}\PYG{p}{(}\PYG{p}{)}
    \PYG{n}{rv} \PYG{o}{=} \PYG{p}{[}\PYG{l+s+sa}{r}\PYG{l+s+s1}{\PYGZsq{}}\PYG{l+s+s1}{\PYGZbs{}}\PYG{l+s+s1}{begin}\PYG{l+s+si}{\PYGZob{}bmatrix\PYGZcb{}}\PYG{l+s+s1}{\PYGZsq{}}\PYG{p}{]}
    \PYG{n}{rv} \PYG{o}{+}\PYG{o}{=} \PYG{p}{[}\PYG{l+s+s1}{\PYGZsq{}}\PYG{l+s+s1}{  }\PYG{l+s+s1}{\PYGZsq{}} \PYG{o}{+} \PYG{l+s+s1}{\PYGZsq{}}\PYG{l+s+s1}{ \PYGZam{} }\PYG{l+s+s1}{\PYGZsq{}}\PYG{o}{.}\PYG{n}{join}\PYG{p}{(}\PYG{n}{l}\PYG{o}{.}\PYG{n}{split}\PYG{p}{(}\PYG{p}{)}\PYG{p}{)} \PYG{o}{+} \PYG{l+s+sa}{r}\PYG{l+s+s1}{\PYGZsq{}}\PYG{l+s+se}{\PYGZbs{}\PYGZbs{}}\PYG{l+s+s1}{\PYGZsq{}} \PYG{k}{for} \PYG{n}{l} \PYG{o+ow}{in} \PYG{n}{lines}\PYG{p}{]}
    \PYG{n}{rv} \PYG{o}{+}\PYG{o}{=}  \PYG{p}{[}\PYG{l+s+sa}{r}\PYG{l+s+s1}{\PYGZsq{}}\PYG{l+s+s1}{\PYGZbs{}}\PYG{l+s+s1}{end}\PYG{l+s+si}{\PYGZob{}bmatrix\PYGZcb{}}\PYG{l+s+s1}{\PYGZsq{}}\PYG{p}{]}
    \PYG{k}{return} \PYG{l+s+s1}{\PYGZsq{}}\PYG{l+s+se}{\PYGZbs{}n}\PYG{l+s+s1}{\PYGZsq{}}\PYG{o}{.}\PYG{n}{join}\PYG{p}{(}\PYG{n}{rv}\PYG{p}{)}

\PYG{n}{E}\PYG{p}{,} \PYG{n}{L}\PYG{p}{,} \PYG{n}{U} \PYG{o}{=} \PYG{n}{lin}\PYG{o}{.}\PYG{n}{lu}\PYG{p}{(}\PYG{n}{A}\PYG{p}{,} \PYG{n}{permute\PYGZus{}l} \PYG{o}{=} \PYG{k+kc}{False}\PYG{p}{)}
\PYG{n}{L} \PYG{o}{=} \PYG{n}{np}\PYG{o}{.}\PYG{n}{matrix}\PYG{p}{(}\PYG{n}{L}\PYG{p}{)}
\PYG{n}{U} \PYG{o}{=} \PYG{n}{np}\PYG{o}{.}\PYG{n}{matrix}\PYG{p}{(}\PYG{n}{U}\PYG{p}{)}
\PYG{n+nb}{print}\PYG{p}{(}\PYG{l+s+s2}{\PYGZdq{}}\PYG{l+s+s2}{L:}\PYG{l+s+s2}{\PYGZdq{}}\PYG{p}{)}
\PYG{n+nb}{print}\PYG{p}{(}\PYG{n}{L}\PYG{p}{)}
\PYG{n+nb}{print}\PYG{p}{(}\PYG{l+s+s2}{\PYGZdq{}}\PYG{l+s+se}{\PYGZbs{}n}\PYG{l+s+s2}{U:}\PYG{l+s+s2}{\PYGZdq{}}\PYG{p}{)}
\PYG{n+nb}{print}\PYG{p}{(}\PYG{n}{U}\PYG{p}{)}
\end{sphinxVerbatim}

\begin{sphinxVerbatim}[commandchars=\\\{\}]
L:
[[1.         0.         0.        ]
 [0.5        1.         0.        ]
 [0.         0.66666667 1.        ]]

U:
[[2.         1.         0.        ]
 [0.         1.5        1.        ]
 [0.         0.         1.33333333]]
\end{sphinxVerbatim}

\sphinxAtStartPar
Let us show, that \(U = DL^T\):
\begin{equation*}
\begin{split}
L^T = 
\begin{bmatrix}
1 & \frac{1}{2} & 0\\
0 & 1 & \frac{2}{3}\\
0 & 0 & 1
\end{bmatrix}
;\hspace{10mm}
D = 
\begin{bmatrix}
2 & 0 & 0\\
0 & \frac{3}{2} & 0\\
0 & 0 & 1
\end{bmatrix}
;\hspace{10mm}
U = DL^T
\end{split}
\end{equation*}
\sphinxAtStartPar
That means, \(A = LU = LDL^T\).

\sphinxAtStartPar
Now, let us solve the system using \(LU\) factorization:
\begin{equation*}
\begin{split}
Ax = b \hspace{3mm} \Rightarrow \hspace{3mm} LUx = b;
\\
\begin{align*}
\begin{cases}
Ux = y;\\
Ly = b
\end{cases}
\end{align*}
\end{split}
\end{equation*}\begin{equation*}
\begin{split}
\begin{bmatrix}
  1 & 0 & 0\\
  \frac{1}{2} & 1 & 0\\
  0 & \frac{2}{3} & 1\\
\end{bmatrix}
\begin{bmatrix}
  y_1\\
  y_2\\
  y_3
\end{bmatrix}
=
\begin{bmatrix}
1\\
2\\
3
\end{bmatrix}
\Rightarrow
\begin{cases}
y_1 = 1\\
y_2 = \frac{3}{2}\\
y_3 = 2
\end{cases}
\end{split}
\end{equation*}\begin{equation*}
\begin{split}
\begin{bmatrix}
2 & 1 & 0\\
0 & \frac{3}{2} & 1\\
0 & 0 & \frac{4}{3}
\end{bmatrix}
\begin{bmatrix}
x_1\\
x_2\\
x_3
\end{bmatrix}
=
\begin{bmatrix}
1\\
\frac{3}{2}\\
2
\end{bmatrix}
\Rightarrow
\begin{cases}
x_1 = \frac{1}{2}\\
x_2 = 0\\
x_3 = \frac{3}{2}
\end{cases}
\end{split}
\end{equation*}\begin{equation*}
\begin{split}
x = 
\begin{bmatrix}
1/2\\
0\\
3/2
\end{bmatrix}
\end{split}
\end{equation*}
\sphinxAtStartPar
Next we solve the system using Jacobi and Gauss\sphinxhyphen{}Seidel methods:

\begin{sphinxVerbatim}[commandchars=\\\{\}]
\PYG{k+kn}{import} \PYG{n+nn}{matplotlib}\PYG{n+nn}{.}\PYG{n+nn}{pyplot} \PYG{k}{as} \PYG{n+nn}{plt}
\PYG{k+kn}{import} \PYG{n+nn}{copy}
\end{sphinxVerbatim}

\begin{sphinxVerbatim}[commandchars=\\\{\}]
\PYG{n}{D} \PYG{o}{=} \PYG{n}{np}\PYG{o}{.}\PYG{n}{matrix}\PYG{p}{(}\PYG{n}{np}\PYG{o}{.}\PYG{n}{diag}\PYG{p}{(}\PYG{n}{np}\PYG{o}{.}\PYG{n}{diag}\PYG{p}{(}\PYG{n}{A}\PYG{p}{)}\PYG{p}{)}\PYG{p}{)}
\PYG{n}{U} \PYG{o}{=} \PYG{n}{np}\PYG{o}{.}\PYG{n}{matrix}\PYG{p}{(}\PYG{n}{np}\PYG{o}{.}\PYG{n}{triu}\PYG{p}{(}\PYG{n}{A}\PYG{o}{\PYGZhy{}}\PYG{n}{D}\PYG{p}{)}\PYG{p}{)}
\PYG{n}{L} \PYG{o}{=} \PYG{n}{np}\PYG{o}{.}\PYG{n}{matrix}\PYG{p}{(}\PYG{n}{np}\PYG{o}{.}\PYG{n}{tril}\PYG{p}{(}\PYG{n}{A}\PYG{o}{\PYGZhy{}}\PYG{n}{D}\PYG{p}{)}\PYG{p}{)}

\PYG{n}{x\PYGZus{}exact} \PYG{o}{=} \PYG{n}{np}\PYG{o}{.}\PYG{n}{matrix}\PYG{p}{(}\PYG{p}{[}\PYG{l+m+mi}{1}\PYG{o}{/}\PYG{l+m+mi}{2}\PYG{p}{,} \PYG{l+m+mi}{0}\PYG{p}{,} \PYG{l+m+mi}{3}\PYG{o}{/}\PYG{l+m+mi}{2}\PYG{p}{]}\PYG{p}{)}\PYG{o}{.}\PYG{n}{T}
\PYG{n}{tol} \PYG{o}{=} \PYG{l+m+mf}{1e\PYGZhy{}8}
\PYG{n}{err} \PYG{o}{=} \PYG{l+m+mi}{1}
\PYG{n}{x\PYGZus{}init} \PYG{o}{=} \PYG{n}{np}\PYG{o}{.}\PYG{n}{matrix}\PYG{p}{(}\PYG{p}{[}\PYG{l+m+mi}{1}\PYG{p}{,} \PYG{l+m+mi}{1}\PYG{p}{,} \PYG{l+m+mi}{1}\PYG{p}{]}\PYG{p}{)}\PYG{o}{.}\PYG{n}{T} \PYG{c+c1}{\PYGZsh{}initial x}
\PYG{n}{b} \PYG{o}{=} \PYG{n}{np}\PYG{o}{.}\PYG{n}{matrix}\PYG{p}{(}\PYG{p}{[}\PYG{l+m+mi}{1}\PYG{p}{,} \PYG{l+m+mi}{2}\PYG{p}{,} \PYG{l+m+mi}{3}\PYG{p}{]}\PYG{p}{)}\PYG{o}{.}\PYG{n}{T}

\PYG{k}{def} \PYG{n+nf}{Jacobi}\PYG{p}{(}\PYG{n}{L}\PYG{p}{,} \PYG{n}{D}\PYG{p}{,} \PYG{n}{U}\PYG{p}{,} \PYG{n}{x\PYGZus{}init}\PYG{p}{,} \PYG{n}{b}\PYG{p}{,} \PYG{n}{err}\PYG{p}{,} \PYG{n}{tol}\PYG{p}{)}\PYG{p}{:}

    \PYG{n}{x} \PYG{o}{=} \PYG{n}{copy}\PYG{o}{.}\PYG{n}{deepcopy}\PYG{p}{(}\PYG{n}{x\PYGZus{}init}\PYG{p}{)}

    \PYG{n}{max\PYGZus{}iters} \PYG{o}{=} \PYG{l+m+mi}{500}

    \PYG{n}{err\PYGZus{}iter} \PYG{o}{=} \PYG{n}{np}\PYG{o}{.}\PYG{n}{array}\PYG{p}{(}\PYG{p}{[}\PYG{p}{]}\PYG{p}{)}
    \PYG{n}{err\PYGZus{}exact\PYGZus{}iter} \PYG{o}{=} \PYG{n}{np}\PYG{o}{.}\PYG{n}{array}\PYG{p}{(}\PYG{p}{[}\PYG{p}{]}\PYG{p}{)}
    

    \PYG{n}{iters} \PYG{o}{=} \PYG{l+m+mi}{0}
    \PYG{k}{while} \PYG{p}{(}\PYG{p}{(}\PYG{n}{err} \PYG{o}{\PYGZgt{}} \PYG{n}{tol}\PYG{p}{)} \PYG{o+ow}{and} \PYG{p}{(}\PYG{n}{iters} \PYG{o}{\PYGZlt{}}\PYG{o}{=} \PYG{n}{max\PYGZus{}iters}\PYG{p}{)}\PYG{p}{)} \PYG{p}{:}
        \PYG{n}{iters} \PYG{o}{=} \PYG{n}{iters} \PYG{o}{+} \PYG{l+m+mi}{1}
        \PYG{n}{bb} \PYG{o}{=} \PYG{n}{b} \PYG{o}{\PYGZhy{}} \PYG{p}{(}\PYG{n}{U} \PYG{o}{+} \PYG{n}{L}\PYG{p}{)}\PYG{o}{*}\PYG{n}{x}
        \PYG{n}{x\PYGZus{}new} \PYG{o}{=} \PYG{n}{np}\PYG{o}{.}\PYG{n}{linalg}\PYG{o}{.}\PYG{n}{solve}\PYG{p}{(}\PYG{n}{D}\PYG{p}{,} \PYG{n}{bb}\PYG{p}{)}
        \PYG{n}{err} \PYG{o}{=} \PYG{n}{np}\PYG{o}{.}\PYG{n}{linalg}\PYG{o}{.}\PYG{n}{norm}\PYG{p}{(}\PYG{n}{x\PYGZus{}new} \PYG{o}{\PYGZhy{}} \PYG{n}{x}\PYG{p}{)}\PYG{o}{/}\PYG{n}{np}\PYG{o}{.}\PYG{n}{linalg}\PYG{o}{.}\PYG{n}{norm}\PYG{p}{(}\PYG{n}{x}\PYG{p}{)}
        \PYG{n}{err\PYGZus{}exact} \PYG{o}{=} \PYG{n}{np}\PYG{o}{.}\PYG{n}{linalg}\PYG{o}{.}\PYG{n}{norm}\PYG{p}{(}\PYG{n}{x\PYGZus{}new} \PYG{o}{\PYGZhy{}} \PYG{n}{x\PYGZus{}exact}\PYG{p}{)}

        \PYG{n}{err\PYGZus{}iter} \PYG{o}{=} \PYG{n}{np}\PYG{o}{.}\PYG{n}{append}\PYG{p}{(}\PYG{n}{err\PYGZus{}iter}\PYG{p}{,} \PYG{n}{err}\PYG{p}{)}
        \PYG{n}{err\PYGZus{}exact\PYGZus{}iter} \PYG{o}{=} \PYG{n}{np}\PYG{o}{.}\PYG{n}{append}\PYG{p}{(}\PYG{n}{err\PYGZus{}exact\PYGZus{}iter}\PYG{p}{,} \PYG{n}{err\PYGZus{}exact}\PYG{p}{)}

        \PYG{n}{x} \PYG{o}{=} \PYG{n}{x\PYGZus{}new}

    \PYG{k}{return} \PYG{n}{err\PYGZus{}iter}\PYG{p}{,} \PYG{n}{err\PYGZus{}exact\PYGZus{}iter}\PYG{p}{,} \PYG{n}{x\PYGZus{}new}\PYG{p}{,} \PYG{n}{iters}

\PYG{k}{def} \PYG{n+nf}{Gauss\PYGZus{}Seidel}\PYG{p}{(}\PYG{n}{L}\PYG{p}{,} \PYG{n}{D}\PYG{p}{,} \PYG{n}{U}\PYG{p}{,} \PYG{n}{x\PYGZus{}init}\PYG{p}{,} \PYG{n}{b}\PYG{p}{,} \PYG{n}{err}\PYG{p}{,} \PYG{n}{tol}\PYG{p}{)}\PYG{p}{:}

    \PYG{n}{x} \PYG{o}{=} \PYG{n}{copy}\PYG{o}{.}\PYG{n}{deepcopy}\PYG{p}{(}\PYG{n}{x\PYGZus{}init}\PYG{p}{)}

    \PYG{n}{max\PYGZus{}iters} \PYG{o}{=} \PYG{l+m+mi}{250}

    \PYG{n}{err\PYGZus{}iter} \PYG{o}{=} \PYG{n}{np}\PYG{o}{.}\PYG{n}{array}\PYG{p}{(}\PYG{p}{[}\PYG{p}{]}\PYG{p}{)}
    \PYG{n}{err\PYGZus{}exact\PYGZus{}iter} \PYG{o}{=} \PYG{n}{np}\PYG{o}{.}\PYG{n}{array}\PYG{p}{(}\PYG{p}{[}\PYG{p}{]}\PYG{p}{)}

    \PYG{n}{iters} \PYG{o}{=} \PYG{l+m+mi}{0}
    \PYG{k}{while} \PYG{p}{(}\PYG{p}{(}\PYG{n}{err} \PYG{o}{\PYGZgt{}} \PYG{n}{tol}\PYG{p}{)} \PYG{o+ow}{and} \PYG{p}{(}\PYG{n}{iters} \PYG{o}{\PYGZlt{}}\PYG{o}{=} \PYG{n}{max\PYGZus{}iters}\PYG{p}{)}\PYG{p}{)}\PYG{p}{:}
        \PYG{n}{iters} \PYG{o}{=} \PYG{n}{iters} \PYG{o}{+} \PYG{l+m+mi}{1}

        \PYG{n}{bb} \PYG{o}{=} \PYG{n}{b} \PYG{o}{\PYGZhy{}} \PYG{n}{U}\PYG{o}{*}\PYG{n}{x}
        \PYG{n}{x\PYGZus{}new} \PYG{o}{=} \PYG{n}{np}\PYG{o}{.}\PYG{n}{linalg}\PYG{o}{.}\PYG{n}{solve}\PYG{p}{(}\PYG{n}{D}\PYG{o}{+}\PYG{n}{L}\PYG{p}{,} \PYG{n}{bb}\PYG{p}{)}
        \PYG{n}{err} \PYG{o}{=} \PYG{n}{np}\PYG{o}{.}\PYG{n}{linalg}\PYG{o}{.}\PYG{n}{norm}\PYG{p}{(}\PYG{n}{x\PYGZus{}new} \PYG{o}{\PYGZhy{}} \PYG{n}{x}\PYG{p}{)}\PYG{o}{/}\PYG{n}{np}\PYG{o}{.}\PYG{n}{linalg}\PYG{o}{.}\PYG{n}{norm}\PYG{p}{(}\PYG{n}{x}\PYG{p}{)}
        \PYG{n}{err\PYGZus{}exact} \PYG{o}{=} \PYG{n}{np}\PYG{o}{.}\PYG{n}{linalg}\PYG{o}{.}\PYG{n}{norm}\PYG{p}{(}\PYG{n}{x\PYGZus{}new} \PYG{o}{\PYGZhy{}} \PYG{n}{x\PYGZus{}exact}\PYG{p}{)}

        \PYG{n}{err\PYGZus{}iter} \PYG{o}{=} \PYG{n}{np}\PYG{o}{.}\PYG{n}{append}\PYG{p}{(}\PYG{n}{err\PYGZus{}iter}\PYG{p}{,} \PYG{n}{err}\PYG{p}{)}
        \PYG{n}{err\PYGZus{}exact\PYGZus{}iter} \PYG{o}{=} \PYG{n}{np}\PYG{o}{.}\PYG{n}{append}\PYG{p}{(}\PYG{n}{err\PYGZus{}exact\PYGZus{}iter}\PYG{p}{,} \PYG{n}{err\PYGZus{}exact}\PYG{p}{)}

        \PYG{n}{x} \PYG{o}{=} \PYG{n}{x\PYGZus{}new}

    \PYG{k}{return} \PYG{n}{err\PYGZus{}iter}\PYG{p}{,} \PYG{n}{err\PYGZus{}exact\PYGZus{}iter}\PYG{p}{,} \PYG{n}{x\PYGZus{}new}\PYG{p}{,} \PYG{n}{iters}
\end{sphinxVerbatim}

\begin{sphinxVerbatim}[commandchars=\\\{\}]
\PYG{o}{\PYGZpc{}}\PYG{k}{config} InlineBackend.figure\PYGZus{}format=\PYGZsq{}svg\PYGZsq{}

\PYG{n}{fig} \PYG{o}{=} \PYG{n}{plt}\PYG{o}{.}\PYG{n}{figure}\PYG{p}{(}\PYG{p}{)}
\PYG{n}{ax} \PYG{o}{=} \PYG{n}{plt}\PYG{o}{.}\PYG{n}{gca}\PYG{p}{(}\PYG{p}{)}

\PYG{n}{errs\PYGZus{}j}\PYG{p}{,} \PYG{n}{ex\PYGZus{}errs\PYGZus{}j}\PYG{p}{,} \PYG{n}{x\PYGZus{}j}\PYG{p}{,} \PYG{n}{i\PYGZus{}j} \PYG{o}{=} \PYG{n}{Jacobi}\PYG{p}{(}\PYG{n}{L}\PYG{p}{,} \PYG{n}{D}\PYG{p}{,} \PYG{n}{U}\PYG{p}{,} \PYG{n}{x\PYGZus{}init}\PYG{p}{,} \PYG{n}{b}\PYG{p}{,} \PYG{n}{err}\PYG{p}{,} \PYG{n}{tol}\PYG{p}{)}
\PYG{n}{plt}\PYG{o}{.}\PYG{n}{plot}\PYG{p}{(}\PYG{n}{errs\PYGZus{}j}\PYG{p}{)}

\PYG{n}{errs\PYGZus{}g}\PYG{p}{,} \PYG{n}{ex\PYGZus{}errs\PYGZus{}g}\PYG{p}{,} \PYG{n}{x\PYGZus{}g}\PYG{p}{,} \PYG{n}{i\PYGZus{}g} \PYG{o}{=} \PYG{n}{Gauss\PYGZus{}Seidel}\PYG{p}{(}\PYG{n}{L}\PYG{p}{,} \PYG{n}{D}\PYG{p}{,} \PYG{n}{U}\PYG{p}{,} \PYG{n}{x\PYGZus{}init}\PYG{p}{,} \PYG{n}{b}\PYG{p}{,} \PYG{n}{err}\PYG{p}{,} \PYG{n}{tol}\PYG{p}{)}
\PYG{n}{plt}\PYG{o}{.}\PYG{n}{plot}\PYG{p}{(}\PYG{n}{errs\PYGZus{}g}\PYG{p}{)}

\PYG{n}{ax}\PYG{o}{.}\PYG{n}{set\PYGZus{}yscale}\PYG{p}{(}\PYG{l+s+s1}{\PYGZsq{}}\PYG{l+s+s1}{log}\PYG{l+s+s1}{\PYGZsq{}}\PYG{p}{)}


\PYG{n}{ax}\PYG{o}{.}\PYG{n}{legend}\PYG{p}{(}\PYG{p}{(}\PYG{l+s+s1}{\PYGZsq{}}\PYG{l+s+s1}{Jacobi}\PYG{l+s+s1}{\PYGZsq{}}\PYG{p}{,} \PYG{l+s+s1}{\PYGZsq{}}\PYG{l+s+s1}{Gauss\PYGZus{}Seidel}\PYG{l+s+s1}{\PYGZsq{}}\PYG{p}{)}\PYG{p}{)}
\PYG{n}{ax}\PYG{o}{.}\PYG{n}{set\PYGZus{}xlabel}\PYG{p}{(}\PYG{l+s+s1}{\PYGZsq{}}\PYG{l+s+s1}{Iterations}\PYG{l+s+s1}{\PYGZsq{}}\PYG{p}{)}
\PYG{n}{ax}\PYG{o}{.}\PYG{n}{set\PYGZus{}ylabel}\PYG{p}{(}\PYG{l+s+s1}{\PYGZsq{}}\PYG{l+s+s1}{Relative error}\PYG{l+s+s1}{\PYGZsq{}}\PYG{p}{)}
\PYG{n}{ax}\PYG{o}{.}\PYG{n}{set\PYGZus{}yticks}\PYG{p}{(}\PYG{p}{[}\PYG{l+m+mf}{1e\PYGZhy{}7}\PYG{p}{,} \PYG{l+m+mf}{1e\PYGZhy{}5}\PYG{p}{,} \PYG{l+m+mf}{1e\PYGZhy{}3}\PYG{p}{,} \PYG{l+m+mf}{1e\PYGZhy{}1}\PYG{p}{]}\PYG{p}{)}
\PYG{n}{ax}\PYG{o}{.}\PYG{n}{tick\PYGZus{}params}\PYG{p}{(}\PYG{n}{axis}\PYG{o}{=}\PYG{l+s+s1}{\PYGZsq{}}\PYG{l+s+s1}{y}\PYG{l+s+s1}{\PYGZsq{}}\PYG{p}{,} \PYG{n}{which}\PYG{o}{=}\PYG{l+s+s1}{\PYGZsq{}}\PYG{l+s+s1}{minor}\PYG{l+s+s1}{\PYGZsq{}}\PYG{p}{)}
\PYG{n}{ax}\PYG{o}{.}\PYG{n}{grid}\PYG{p}{(}\PYG{n}{which}\PYG{o}{=}\PYG{l+s+s1}{\PYGZsq{}}\PYG{l+s+s1}{both}\PYG{l+s+s1}{\PYGZsq{}}\PYG{p}{)}
\end{sphinxVerbatim}

\begin{sphinxVerbatim}[commandchars=\\\{\}]
\PYGZlt{}Figure size 432x288 with 1 Axes\PYGZgt{}
\end{sphinxVerbatim}

\sphinxAtStartPar
As we see, it took \(26\) iterations for Gauss\sphinxhyphen{}Seidel method to reach the target tolerance of \(10^{-8}\), while Jacobi method required \(54\) iterations.

\sphinxAtStartPar
This can be explained by the following: in Gauss\sphinxhyphen{}Seidel, as soon as we acquire a new iteration of a vector \(x\) component \(x_i^{(k+1)}\), we instantly utilize this updated value in the computation of the following components: \(x_i^{(k+1)} = f(x_1^{(k+1)}, ..., x_{i-1}^{(k+1)}, x_{i+1}^{(k)}, ..., x_n^{(k)})\). In Jacobi, we calculate the new vector \(x^{k+1}\) relying solely on the result of the previous iteration: \(x_i^{k+1} = f(x_j^{(k)}),\hspace{3mm}j \neq i\).


\section{Problem 2}
\label{\detokenize{notebooks/ProblemSet2:problem-2}}
\sphinxAtStartPar
Factor these two matrices \(A\) into \(S\Lambda S^{-1}\):
\begin{equation*}
\begin{split}
A1 = 
\begin{bmatrix}
1 & 2\\
0 & 3
\end{bmatrix}
,\hspace{3mm}
A2 =
\begin{bmatrix}
1 & 2\\
0 & 3
\end{bmatrix}
\end{split}
\end{equation*}
\sphinxAtStartPar
Using that factorization, find for both: (a) \(A^3\); (b) \(A^{-1}\).


\subsection{Solution}
\label{\detokenize{notebooks/ProblemSet2:id1}}
\sphinxAtStartPar
Firstly, we find the eigenvalues and eigenvectors:
\begin{equation*}
\begin{split}
\begin{bmatrix}
1-\lambda & 2\\
0 & 3-\lambda
\end{bmatrix}
 = 0
\end{split}
\end{equation*}
\sphinxAtStartPar
By performing simple calculations by hand, we obtain:
\begin{equation*}
\begin{split}
h_1 = 
\begin{bmatrix}
1\\
0
\end{bmatrix}
,\hspace{3mm}
\lambda_1 = 1
\\
h_2 =
\frac{\sqrt{2}}{2}
\begin{bmatrix}
1\\
1
\end{bmatrix}
,\hspace{3mm}
\lambda_2 = 3
\end{split}
\end{equation*}
\begin{sphinxadmonition}{note}{Note:}
\sphinxAtStartPar
We normalized the eigenvectors
\end{sphinxadmonition}

\sphinxAtStartPar
Now let’s perform a check with \sphinxstyleemphasis{numpy}:

\begin{sphinxVerbatim}[commandchars=\\\{\}]
\PYG{n}{A1} \PYG{o}{=} \PYG{n}{np}\PYG{o}{.}\PYG{n}{matrix}\PYG{p}{(}\PYG{p}{[}
    \PYG{p}{[}\PYG{l+m+mi}{1}\PYG{p}{,} \PYG{l+m+mi}{2}\PYG{p}{]}\PYG{p}{,}
    \PYG{p}{[}\PYG{l+m+mi}{0}\PYG{p}{,} \PYG{l+m+mi}{3}\PYG{p}{]}
\PYG{p}{]}\PYG{p}{)}

\PYG{n}{val1}\PYG{p}{,} \PYG{n}{vec1} \PYG{o}{=} \PYG{n}{np}\PYG{o}{.}\PYG{n}{linalg}\PYG{o}{.}\PYG{n}{eig}\PYG{p}{(}\PYG{n}{A1}\PYG{p}{)}
\PYG{n+nb}{print}\PYG{p}{(}\PYG{l+s+s2}{\PYGZdq{}}\PYG{l+s+s2}{E\PYGZus{}values A1:}\PYG{l+s+s2}{\PYGZdq{}}\PYG{p}{)}
\PYG{n+nb}{print}\PYG{p}{(}\PYG{n}{val1}\PYG{p}{)}
\PYG{n+nb}{print}\PYG{p}{(}\PYG{l+s+s2}{\PYGZdq{}}\PYG{l+s+s2}{E\PYGZus{}vectors A1:}\PYG{l+s+s2}{\PYGZdq{}}\PYG{p}{)}
\PYG{n+nb}{print}\PYG{p}{(}\PYG{n}{vec1}\PYG{p}{)}
\end{sphinxVerbatim}

\begin{sphinxVerbatim}[commandchars=\\\{\}]
E\PYGZus{}values A1:
[1. 3.]
E\PYGZus{}vectors A1:
[[1.         0.70710678]
 [0.         0.70710678]]
\end{sphinxVerbatim}

\sphinxAtStartPar
As we have our vectors and values, we can construct \(\Lambda\) and \(S\), \(S^{-1}\) matrices:

\begin{sphinxVerbatim}[commandchars=\\\{\}]
\PYG{n}{Lambda} \PYG{o}{=} \PYG{n}{np}\PYG{o}{.}\PYG{n}{matrix}\PYG{p}{(}\PYG{n}{np}\PYG{o}{.}\PYG{n}{diag}\PYG{p}{(}\PYG{n}{val1}\PYG{p}{)}\PYG{p}{)}
\PYG{n}{S} \PYG{o}{=} \PYG{n}{vec1}
\PYG{n}{Si} \PYG{o}{=} \PYG{n}{np}\PYG{o}{.}\PYG{n}{linalg}\PYG{o}{.}\PYG{n}{inv}\PYG{p}{(}\PYG{n}{S}\PYG{p}{)}

\PYG{n+nb}{print}\PYG{p}{(}\PYG{l+s+s2}{\PYGZdq{}}\PYG{l+s+s2}{S:}\PYG{l+s+s2}{\PYGZdq{}}\PYG{p}{)}
\PYG{n+nb}{print}\PYG{p}{(}\PYG{n}{S}\PYG{p}{)}
\PYG{n+nb}{print}\PYG{p}{(}\PYG{l+s+s2}{\PYGZdq{}}\PYG{l+s+s2}{Lambda:}\PYG{l+s+s2}{\PYGZdq{}}\PYG{p}{)}
\PYG{n+nb}{print}\PYG{p}{(}\PYG{n}{Lambda}\PYG{p}{)}
\PYG{n+nb}{print}\PYG{p}{(}\PYG{l+s+s2}{\PYGZdq{}}\PYG{l+s+s2}{S\PYGZus{}inverse:}\PYG{l+s+s2}{\PYGZdq{}}\PYG{p}{)}
\PYG{n+nb}{print}\PYG{p}{(}\PYG{n}{Si}\PYG{p}{)}
\end{sphinxVerbatim}

\begin{sphinxVerbatim}[commandchars=\\\{\}]
S:
[[1.         0.70710678]
 [0.         0.70710678]]
Lambda:
[[1. 0.]
 [0. 3.]]
S\PYGZus{}inverse:
[[ 1.         \PYGZhy{}1.        ]
 [ 0.          1.41421356]]
\end{sphinxVerbatim}
\begin{equation*}
\begin{split}S = 
\begin{bmatrix}
1 & \sqrt{2}/2\\
0 & \sqrt{2}/{2}
\end{bmatrix}
;\hspace{3mm}
\Lambda = 
\begin{bmatrix}
1 & 0\\
0 & 3
\end{bmatrix}
;\hspace{3mm}
S^{-1} = 
\begin{bmatrix}
1 & -1\\
0 & 2/\sqrt{2}
\end{bmatrix}
\end{split}
\end{equation*}
\sphinxAtStartPar
From now, we can fing the \(A^3\) powered matrix by simply multiplying the decomposition:
\begin{equation*}
\begin{split}
A^3 = S\Lambda S^{-1} S \Lambda S^{-1} S \Lambda S^{-1} =
\\
\hspace{1mm}
\\
= S \Lambda^3 S^{-1}.
\end{split}
\end{equation*}
\sphinxAtStartPar
And for \(\Lambda\) it is easy to power because it is a diagonal matrix.
\begin{equation*}
\begin{split}
\Lambda ^3 =
\begin{bmatrix}
1 & 0\\
0 & 9
\end{bmatrix}
,
\\
\hspace{1mm}
\\
S\Lambda ^3 S^{-1} = 
\begin{bmatrix}
1 & 8\\
0 & 9
\end{bmatrix}
\end{split}
\end{equation*}
\sphinxAtStartPar
For inverse matrix:
\begin{equation*}
\begin{split}
S\Lambda S^{-1} A_1^{-1} = E, \Rightarrow A_1^{-1} = S^{-1} \Lambda ^{-1} S
\end{split}
\end{equation*}
\sphinxAtStartPar
We already have \(S\) and \(S^{-1}\), and for diagonal \(\Lambda\) the inverse matrix contains the inverse diagonal elements of \(\Lambda\):
\begin{equation*}
\begin{split}
\Lambda ^{-1}= 
\begin{bmatrix}
1 & 0\\
0 & 1/3
\end{bmatrix}
\end{split}
\end{equation*}
\sphinxAtStartPar
So we easily find \(A_1^{-1}\):
\begin{equation*}
\begin{split}
A_1^{-1} = 
\begin{bmatrix}
1 & \sqrt{2}{3}\\
0 & 1/3
\end{bmatrix}
\end{split}
\end{equation*}
\sphinxAtStartPar
Now let’s look at the second matrix \(A_2\). Instantly we notice it is a rank\sphinxhyphen{}1 matrix, thus, \(A_2^{-1}\) matrix doesn’t exist.
\begin{equation*}
\begin{split}
h_1 = 
\begin{bmatrix}
1\\
-1
\end{bmatrix}
,\hspace{3mm}
\lambda_1 = 0
\\
\hspace{1mm}
\\
h_2 =
\begin{bmatrix}
1\\
3
\end{bmatrix}
,\hspace{3mm}
\lambda_2 = 4
\end{split}
\end{equation*}
\sphinxAtStartPar
It’s eigenvectors are non\sphinxhyphen{}collinear and form a basis in 2\sphinxhyphen{}dimensional space. Thus we can perform the factorization.
\begin{equation*}
\begin{split}
S = 
\begin{bmatrix}
1 & 1\\
-1 & 3
\end{bmatrix}
;\hspace{3mm}
\Lambda = 
\begin{bmatrix}
0 & 0\\
0 & 4
\end{bmatrix}
;\hspace{3mm}
S^{-1} =  
\begin{bmatrix}
3/4 & -1/4\\
1/4 & 1/4
\end{bmatrix}
\end{split}
\end{equation*}
\sphinxAtStartPar
For \(A_2^3\):
\begin{equation*}
\begin{split}
A_2^3 = 
\begin{bmatrix}
64 & 64\\
192 & 192
\end{bmatrix}
.
\end{split}
\end{equation*}

\section{Problem 3}
\label{\detokenize{notebooks/ProblemSet2:problem-3}}
\sphinxAtStartPar
Given a system \(Ax = b\) with
\begin{equation*}
\begin{split}
A = 
\begin{bmatrix}
  1 & -1 & -3\\
  2 & 3 & 4\\
  -2 & 1 & 4\\
\end{bmatrix}
,\hspace{3mm}
b = 
\begin{bmatrix}
  3\\
  a\\
  -1\\
\end{bmatrix}
,
\end{split}
\end{equation*}
\sphinxAtStartPar
for which \(a\) there is a solution? Find the general solution of the system for that \(a\).


\subsection{Solution}
\label{\detokenize{notebooks/ProblemSet2:id2}}
\sphinxAtStartPar
Let’s check the matrix’ rank:

\begin{sphinxVerbatim}[commandchars=\\\{\}]
\PYG{n}{a} \PYG{o}{=} \PYG{n}{abc}\PYG{o}{.}\PYG{n}{symbols}\PYG{p}{(}\PYG{l+s+s1}{\PYGZsq{}}\PYG{l+s+s1}{a}\PYG{l+s+s1}{\PYGZsq{}}\PYG{p}{)}

\PYG{n}{A} \PYG{o}{=} \PYG{n}{sp}\PYG{o}{.}\PYG{n}{Matrix}\PYG{p}{(}\PYG{p}{[}
    \PYG{p}{[}\PYG{l+m+mi}{1}\PYG{p}{,} \PYG{o}{\PYGZhy{}}\PYG{l+m+mi}{1}\PYG{p}{,} \PYG{o}{\PYGZhy{}}\PYG{l+m+mi}{3}\PYG{p}{]}\PYG{p}{,}
    \PYG{p}{[}\PYG{l+m+mi}{2}\PYG{p}{,} \PYG{l+m+mi}{3}\PYG{p}{,} \PYG{l+m+mi}{4}\PYG{p}{]}\PYG{p}{,}
    \PYG{p}{[}\PYG{o}{\PYGZhy{}}\PYG{l+m+mi}{2}\PYG{p}{,} \PYG{l+m+mi}{1}\PYG{p}{,} \PYG{l+m+mi}{4}\PYG{p}{]}
\PYG{p}{]}\PYG{p}{)}

\PYG{n}{b} \PYG{o}{=} \PYG{n}{sp}\PYG{o}{.}\PYG{n}{Matrix}\PYG{p}{(}\PYG{p}{[}
    \PYG{l+m+mi}{3}\PYG{p}{,} \PYG{n}{a}\PYG{p}{,} \PYG{o}{\PYGZhy{}}\PYG{l+m+mi}{1}\PYG{p}{]}
\PYG{p}{)}

\PYG{n+nb}{print}\PYG{p}{(}\PYG{l+s+s1}{\PYGZsq{}}\PYG{l+s+s1}{Rank: }\PYG{l+s+si}{\PYGZob{}\PYGZcb{}}\PYG{l+s+s1}{\PYGZsq{}}\PYG{o}{.}\PYG{n}{format}\PYG{p}{(}\PYG{n}{A}\PYG{o}{.}\PYG{n}{rank}\PYG{p}{(}\PYG{p}{)}\PYG{p}{)}\PYG{p}{)}
\PYG{n}{A}\PYG{o}{.}\PYG{n}{rref}\PYG{p}{(}\PYG{p}{)}\PYG{p}{[}\PYG{l+m+mi}{0}\PYG{p}{]}
\end{sphinxVerbatim}

\begin{sphinxVerbatim}[commandchars=\\\{\}]
Rank: 2
\end{sphinxVerbatim}
\begin{equation*}
\begin{split}\displaystyle \left[\begin{matrix}1 & 0 & -1\\0 & 1 & 2\\0 & 0 & 0\end{matrix}\right]\end{split}
\end{equation*}
\sphinxAtStartPar
This matrix is a rank\sphinxhyphen{}2 matrix. Let’s find it’s left nullspace, write down the solvability condition and find the appropriate \(a\).
Starting with the left null\sphinxhyphen{}space:

\begin{sphinxVerbatim}[commandchars=\\\{\}]
\PYG{n}{y} \PYG{o}{=} \PYG{n}{A}\PYG{o}{.}\PYG{n}{T}\PYG{o}{.}\PYG{n}{nullspace}\PYG{p}{(}\PYG{p}{)}\PYG{p}{[}\PYG{l+m+mi}{0}\PYG{p}{]}
\PYG{n}{y}
\end{sphinxVerbatim}
\begin{equation*}
\begin{split}\displaystyle \left[\begin{matrix}\frac{8}{5}\\\frac{1}{5}\\1\end{matrix}\right]\end{split}
\end{equation*}
\sphinxAtStartPar
We want to fulfill the following condition:
\begin{equation*}
\begin{split}
y^Tb = 0
\end{split}
\end{equation*}
\begin{sphinxVerbatim}[commandchars=\\\{\}]
\PYG{n}{ans} \PYG{o}{=} \PYG{n}{sp}\PYG{o}{.}\PYG{n}{solve}\PYG{p}{(}\PYG{n}{y}\PYG{o}{.}\PYG{n}{T}\PYG{o}{*}\PYG{n}{b}\PYG{p}{,} \PYG{n}{a}\PYG{p}{)}
\PYG{n}{ans}
\end{sphinxVerbatim}

\begin{sphinxVerbatim}[commandchars=\\\{\}]
\PYGZob{}a: \PYGZhy{}19\PYGZcb{}
\end{sphinxVerbatim}

\sphinxAtStartPar
As we see, the system is solvable with \(a = -19\). Let us perform some check:

\begin{sphinxVerbatim}[commandchars=\\\{\}]
\PYG{n}{bs} \PYG{o}{=} \PYG{n}{b}\PYG{o}{.}\PYG{n}{subs}\PYG{p}{(}\PYG{n}{a}\PYG{p}{,} \PYG{o}{\PYGZhy{}}\PYG{l+m+mi}{19}\PYG{p}{)}
\PYG{n}{y}\PYG{o}{.}\PYG{n}{T}\PYG{o}{*}\PYG{n}{bs}
\end{sphinxVerbatim}
\begin{equation*}
\begin{split}\displaystyle \left[\begin{matrix}0\end{matrix}\right]\end{split}
\end{equation*}
\sphinxAtStartPar
Now as we have our vector \(b\) with wich the system is solvable, we may find the general solution for the system:

\begin{sphinxVerbatim}[commandchars=\\\{\}]
\PYG{n}{a}\PYG{p}{,} \PYG{n}{b}\PYG{p}{,} \PYG{n}{c} \PYG{o}{=} \PYG{n}{abc}\PYG{o}{.}\PYG{n}{symbols}\PYG{p}{(}\PYG{l+s+s1}{\PYGZsq{}}\PYG{l+s+s1}{a b c}\PYG{l+s+s1}{\PYGZsq{}}\PYG{p}{)}
\PYG{n}{system} \PYG{o}{=} \PYG{n}{A}\PYG{p}{,} \PYG{n}{bs}
\PYG{n}{sol} \PYG{o}{=} \PYG{n}{sp}\PYG{o}{.}\PYG{n}{linsolve}\PYG{p}{(}\PYG{p}{(}\PYG{n}{A}\PYG{p}{,} \PYG{n}{bs}\PYG{p}{)}\PYG{p}{,} \PYG{n}{a}\PYG{p}{,} \PYG{n}{b}\PYG{p}{,} \PYG{n}{c}\PYG{p}{)}\PYG{p}{;} \PYG{n}{sol}
\end{sphinxVerbatim}
\begin{equation*}
\begin{split}\displaystyle \left\{\left( c - 2, \  - 2 c - 5, \  c\right)\right\}\end{split}
\end{equation*}
\sphinxAtStartPar
With \(c \in \mathbb{R}\) we get our solution:
\begin{equation*}
\begin{split}
x =
\begin{bmatrix}
-2\\
-5\\
0
\end{bmatrix}
+
\begin{bmatrix}
1\\
-2\\
1
\end{bmatrix}
\cdot
c.
\end{split}
\end{equation*}

\section{Problem 4}
\label{\detokenize{notebooks/ProblemSet2:problem-4}}






\renewcommand{\indexname}{Index}
\printindex
\end{document}